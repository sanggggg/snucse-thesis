\chapter{관련 연구}\label{chap:background}

본 연구는 'AI 협업 데이터 분석'과 '채팅 히스토리 정성 평가'라는 두 가지 주요 영역의 기존 연구들을 기반으로 한다.

\section{AI 협업 데이터 분석 툴링}\label{sec:ai_collab_tools}

AI와의 협업 개발 형태가 보편화되면서, 이러한 상호작용 데이터를 회고하고 분석하려는 시도들이 등장하고 있다. 대표적으로 ryoppippi/ccusage\cite{ccusage}나 sculptdotfun/viberank\cite{viberank}와 같은 오픈소스 프로젝트들이 있다.

\textbf{ryoppippi/ccusage:} 주로 AI 사용에 따른 비용(Cost)과 토큰(Token) 사용량을 추적하고 시각화하는 데 중점을 둔다. 이는 조직이나 개인이 AI 활용에 드는 비용을 관리하는 데 유용하다.

\textbf{sculptdotfun/viberank:} AI 사용에 따른 토큰(Token) 사용량을 기반으로, 사용량을 달러 단위의 비용으로 변환하여 보여주는 대시보드 사이트이다.

\begin{figure}[H]
    \centering
    \begin{subfigure}[b]{0.48\textwidth}
        \centering
        \includegraphics[width=\textwidth]{figures/ccusage.png}
        \caption{ccusage}
        \label{fig:ccusage}
    \end{subfigure}
    \hfill % or \quad
    \begin{subfigure}[b]{0.48\textwidth}
        \centering
        \includegraphics[width=\textwidth]{figures/viberank.png}
        \caption{viberank}
        \label{fig:viberank}
    \end{subfigure}
    \caption{ccusage 사용자 화면 및 viberank 사용자 화면}
    \label{fig:ccusage_viberank}
\end{figure}

이러한 도구들은 AI 협업의 특정 단면(주로 비용 및 사용량)을 정량적으로 파악하는 데 유용성을 가지나, 사용자가 AI에 얼마나 명확하게 요구사항을 전달했는지, AI의 제안을 얼마나 비판적으로 수용하고 개선했는지와 같은 상호작용의 '질'을 평가하는 데는 명확한 한계가 있다.

\section{채팅 히스토리 정성 평가 방법론}\label{sec:chat_eval}

AI와의 채팅 히스토리를 정성적으로 평가하려는 연구도 활발히 진행되고 있다. 그중 본 연구와 밀접하게 관련된 것은 'SPUR'\cite{spur} 방법론이다.

\textbf{SPUR (Supervised Prompting for User satisfaction Rubrics):} 이 연구는 사용자의 '호/불호'가 레이블링된 정성적인 채팅 히스토리 데이터셋을 활용한다. 지도 학습(Supervised Learning) 형태를 통해, 사용자가 특정 채팅 내역을 선호하거나 선호하지 않는 이유가 되는 다양한 측면의 피처(루브릭)를 모델이 스스로 학습하여 뽑아내도록 한다. 이렇게 학습된 모델은 새로운 채팅 히스토리가 입력되었을 때, 해당 상호작용에 대한 사용자의 잠재적인 호/불호를 예측하고 그 근거를 제시한다.

\begin{figure}[H]
    \centering
    \includegraphics[width=\textwidth]{figures/spur_approach.png}
    \caption{SPUR 방법론의 지도학습 과정 / 평가 과정}
    \label{fig:spur}
\end{figure}

SPUR는 LLM을 활용해 채팅의 정성적 품질을 평가하는 루브릭을 도출할 수 있다는 가능성을 보여주었다. 하지만 이는 사용자의 '선호도' 예측에 초점을 맞추고 있다. 본 연구는 여기서 더 나아가, 선호도를 넘어 다양한 정량 지표를 바탕으로 사용자의 '활용 능력'을 판단하는 루브릭을 개발하고 이를 점수화하는 시스템을 구축하고자 한다.
