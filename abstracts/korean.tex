\chapter*{국문초록}
\addcontentsline{toc}{chapter}{국문초록}

\textbf{주요어:} AI 코딩 에이전트, AI 활용 능력, 지표 개발, RetroChat, LLM-as-a-judge, 루브릭

AI 코딩 에이전트가 소프트웨어 개발의 새로운 패러다임으로 자리 잡음에 따라, 사용자가 얼마나 효과적으로 AI와 협업하는지 측정할 필요성이 대두되고 있다. 기존의 분석 도구들은 비용이나 토큰 사용량과 같은 단순한 정량적 지표에 머무르는 한계가 있었다. 본 연구는 이러한 한계를 극복하기 위해 두 가지 핵심 요소를 제안한다.

첫째, 다양한 상용 AI 코딩 에이전트의 채팅 로그를 다면적으로 수집하고 분석하는 오픈소스 툴킷 'RetroChat'을 개발한다.

둘째, 수집된 채팅 히스토리 원천 데이터를 기반으로, 'LLM-as-a-judge' 방법론을 활용하여 사용자의 AI 상호작용 품질을 평가하는 시스템을 구축한다. 본 연구에서는 토큰 효율성과 같은 객관적 지표를 기준으로 우수 세션을 선별하고, LLM을 통해 해당 세션들의 성공 요인을 학습하여 평가 루브릭(rubric)을 자동으로 도출하는 지도학습 기반의 파이프라인을 제안한다. 이 시스템은 사용자의 'AI 활용 능력(AI Proficiency)'을 루브릭에 따라 다각도로 점수화함으로써, 개발자 교육 및 AI 협업 프로세스 개선에 기여하는 것을 목표로 한다.
